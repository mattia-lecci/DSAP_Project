\section{Experiment Setup}
\label{sec:setup}

Our tests were split in two parts: the first was about recognizing the chord from a single chord track, the second one was about recognizing the chords of full songs.\\
%
For the first part, we used the dataset \texttt{jim2012Chords} \cite{jim2012Chords} created for their paper \cite{JimChordsPaper}. It's composed of a total of over $2.000$ recordings of 10 guitar chords (both major and minor triads). Four different guitars are used, as well as piano, violin and accordion. Some tracks were recorded in an anechoic chamber, some other in a noisy environment. For each one of the tracks we obtain a 12-dimensional feature vector using \textit{Chroma Toolbox} and taking the maximum value for each note. Looking at Fig.~\ref{fig:CENSexample}, the idea is that we are interested in high values of specific groups of notes in order to guess the chord and the maximum value is a simple way of doing this. It's biggest disadvantage is that impulsive noisy environment can easily mask the useful information introducing high values in the wrong notes. Finally, we run the different classification methods on these feature vectors randomly splitting training/test data into 70\%/30\% (i.e. $r_{test}=0.3$).\\
%
For the second part we used as dataset \textit{The Beatles}' discography, which has been professionally transcribed by \textit{C. Harte} \cite{HartePaper}\cite{HarteThesis}. In his work, Harte performed a very detailed transcription, obtaining a wide set of chords. To reduce computational cost we mapped this wide set of $66$ chords into the $12$ notes of the chromatic scale in the major and minor version. We used a total of 150 songs from this dataset, splitting them into training and test sets according to $r_{test}\in \{0.1,0.2,0.3\}$. From now on, we indicate our our songs dataset with $\mathcal{D}_{Beatles}$. \\
%
Every song contained in $\mathcal{D}_{Beatles}$ was processed according to one of the techniques offered by \textit{Chroma toolbox}. We set toolbox parameters so that we didn't get an exaggerated number of frames per song. In \textit{results} section we will see that this choice could make outputs weaker. After features extraction was completed, we obtained a sequence of \textit{frame-features} for each song. The length of the sequence depended both on the song duration and the chosen featuring process (CENS,CLP,CRP). We divided the periods computed by Harte in sub-periods with equal time-length to the frames computed with Chroma toolbox. Therefore for each song we obtained a sequence of \textit{frame-labels} with equal length to the sequence of \textit{frame-features}; from now on we indicate the \textit{frame-labels} sequence for a generic song with $\mathcal{L}_{Harte}$. Finally, we noticed that almost all the songs presented periods with no sounds, which were marked with the `$N$' label. Obviously `$N$' frames were out of our interest and thus discarded. \\
%
At the end of the \textit{pre-processing} phase we obtained two sets of songs $(\mathcal{D}_{train},\mathcal{D}_{test})$, in which every songs is assigned to both a sequence of \textit{frame-features} and a sequence of \textit{frame-labels}. The data contained in $\mathcal{D}_{train}$ were then used to train the MC-SVM and subsequently build the HMM. MC-SVM training was achieved assuming one chord per frame and using as kernel the \textit{polynomial} one with degree $3$. HMM training was achieved testing the data contained in $\mathcal{D}_{train}$ with the MC-SVM previously built. The resulting errors allowed us to establish values for the \textit{emission probabilities}. Analyzing directly all $\mathcal{L}_{Harte}$, \textit{transition probabilities} and \textit{initial probabilities} values were computed. \\
%
Once all our methods were trained, we took the \textit{frame-features} of $\mathcal{D}_{test}$ and we processed them first only with MC-SVM methods, then also with HMM. Therefore for each song we obtained two different outputs, that we call $\mathcal{L}_{SVM}$ and $\mathcal{L}_{HMM}$. $\mathcal{L}_{SVM}$ corresponds to the \textit{frame-labels} sequence produced by the MC-SVM. In this process every frame was assigned to a chord in a memoryless fashion. $\mathcal{L}_{HMM}$ corresponds to the \textit{frame-labels} sequence produced by the HMM, which took MC-SVM's output and tried to improve it through \textit{Viterbi algorithm}. In particular HMM took into account the whole sequence of chords that composes the song.
