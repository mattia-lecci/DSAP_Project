\subsection{Gaussian Mixture Models (GMMs)}
\label{subsec:gmm}

\textit{Gaussian Mixture Models} are a statistic technique used to identify classes within an overall population. At the end of the process, each class is assigned to a \textit{Multivariate Gaussian Distribution} (MGD).  As widely known, MGDs are defined only through its \textit{mean vector} and a \textit{covariance matrix}. We notice that given a MGD, it is very simple to evaluate the probability distribution function for that point. We just need to implement the \textit{Mahalanobis distance} which measures how many standard deviations the selected object is far from the distribution mean, taking into account the covariance matrix. \\
%
MATLAB defines a class called \texttt{gmdistribution} which allows you to easily create an object which contains one or more MGDs. MATLAB also offers the method \texttt{mahal} to compute the Mahalanobis distance between an object $X$ and a GMM.