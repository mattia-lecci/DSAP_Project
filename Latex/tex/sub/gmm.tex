\subsection{Gaussian Mixture Models (GMMs)}
\label{subsec:gmm}

\textit{Gaussian Mixture Model} represents a statistic technique used to individue subpopolations within an overall population. GMMs can be used in a supervised or not-supervised manner and in both the case they describe each subpopulation with a \gls{mgd}.  \gls{mgd} represents the best known and used form of multivariate distribution and is defined only by a mean vector and a covariance matrix. We notice that given a \gls{mgd} it is very simple determine the probability of that an object is included by the population described by the distribution. This can be done using the so-called \textit{Mahalanobis distance} which measures how many standard deviations the considered object is distant from the center of the distribution. \\
%
MATLAB defines a class called \textit{gmdistribution} which allows to construct an object that contain one or more \gls{mgd}s. This object can be built either by adding every distributions by ourself (supervised method) or fitting the entire population to a chosen number of distributions (not-supervised method). MATLAB offers also the method \textit{mahal} to compute the Mahalanobis distance between an object $X$ and a GMM. \\
%
GMMs represent a very simple and effective technique, which can be easily implemented with the pre-established MATLAB methods. In our work we focus on the supervised approach of GMM, creating the singles subpopulations one at time during the so-called \textit{training phase}.
