\subsection{Template-based chord recognition}
\label{subsec:templates}

Template matching consists of two steps: first, the templates are computed and second, the distance between the extracted features and the templates is calculated. In this project two types of templates are used: the \textit{binary templates} and the \textit{harmonic templates}. Binary templates are a simple binary representation of the chords (0 if the note is not present, $1/\sqrt{3}$ if the note is present). In order to create more accurate templates, the harmonics of each chord's note are also taken in count for the harmonic templates.

Adopting techniques taken from \cite{gomez2006tonal},\cite{oudre2009chord}, the first 6 harmonics of each note where used, exponentially attenuating each successive harmonic of each note (i.e. multiplying the harmonics by $[1, s, s^2, s^3, s^4, s^5]$, with the suggested factor $s=0.6$) and finally adding everything up and normalizing the result.