\subsection{Template-based chord recognition}
\label{subsec:templates}

After computing the Chroma and having some labelled results the classification of them can begin. One method is to used template matching techniques. This procedure consist on two steps. In the first step the templates are computed and in the second the distance between the extracted features and the templates is calculated. In this project two Templates are used: The Binary Templates and the Harmonic Templates. The Binary Templates are a binary $([1,0]-Values)$ representation of the Chords. In order to create more acculturate templates the harmonics of each chord-note are also taken in count and as a result the Harmonic Templates are implemented.  “\textit{The sound of a musical tone can be regarded as a superposition of harmonics or overtones - whose frequencies differ by an integer multiple from a certain fundamental frequency.}” \cite{muller2007information}
\\
In order to create the Binary Templates, 12-Rows Vector where created for each minor and major triad. As a result a $12x24$ Matrix is created. The Vectors describe in a binary way each Chord, for example the C-major chord has a 1-Value at the $\{C,E,G\}$ Notes that describes it, all other are zeros.\\
An improvement of this method is to use also the harmonics of each note. Following some other researches\cite{gomez2006tonal}\cite{oudre2009chord} the 6 harmonics of each note where used, with an attenuation factor of $s=0.6$ and a increasing of this factor by one magnitude per harmonic $[s, s^2, s^3, s^4, s^5]$. Following this procedure for all 3 Note components of each Chord and adding it, the resulting $(12x24)$-Harmonic Template Matrix is achieved. 