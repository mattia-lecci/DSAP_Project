\subsection{Template-based chord recognition}
\label{subsec:templates}

Template matching consists in two steps. First we compute the template for each possible chord, then we calculate the distance between the extracted features and all the templates. We assign to the considered feature the chord with which presents a lower distance. In this project two types of templates are used: the \textit{binary templates} and the \textit{harmonic templates}. Binary templates are a simple binary representation of the chords (0 if the note is not present, $1/\sqrt{3}$ if the note is present). \\
%
Harmonic templates represent a more accurate choice. Adopting the techniques proposed by \cite{gomez2006tonal} and \cite{oudre2009chord}, the first 6 harmonics of each note are used. Doing so we exponentially attenuate each successive harmonic of each note (i.e. multiplying the harmonics by $[1, s, s^2, s^3, s^4, s^5]$, with the suggested factor $s=0.6$). After that the outputs of each harmonic are addede together into a single template, which is at the end normalized.
