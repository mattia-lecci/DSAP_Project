\section{Introduction}
\label{sec:intro}

A chord is defined as two or more notes sounding simultaneously \cite{benward2014music}. This is normally achieved by playing different notes at the same time; in others cases like arpeggios and broken chords the notes are played separately and successively. Given a music piece the chords and their progression define the harmonic content of the piece itslef \cite{papadopoulos2007large}. Analyze chord tonality is one of the most important task in both \textit{Chord Recognition} and \textit{Song Identification} \cite{lee2006identifying}.\\
%
The most commonly used chords in music are \textit{triads}, wich coincide with groups of three notes (root, third and perfect fifth). One of the most difficult problem of \textit{Chord Recognition} is manage chords played in different ocatves. To overcome this issue it is possible to use \textit{chromas} instead of distinct notes \cite{bartsch2005audio}. Chords can be classified as \textit{minor}, \textit{major}, \textit{dimished} and \textit{augmented}, according to the musical interval of the notes that costituite them. The name of a chord is driven from its root note, usually the lowest note. Considering triad as example, the second note is called third, because it's the third chroma from the root note, and the third note is called fifth. The chord is called major (minor) if the third interval is major (minor), i.e. 4 (3) semitones away from the root. We notice that some chords present inversions, if the root note is not in the lowest position. All this peculiarities make \textit{Chord Recognition} a very complex task to be achieved. \\
%
In the first part of our work we focus on triads analysis and we try to perform \textit{Chord Recognition} using sound containing a single chord realized by a single instrument. In the second part we deal with a much complex scenario, since we try to perform \textit{Chord Recognition} using full songs that are composed by several instruments and also vocal parts.
