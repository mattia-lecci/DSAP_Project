\section{Introduction}
\label{sec:intro}

A chord is defined as two or more notes sounding simultaneously \cite{benward2014music}. This is normally achieved by playing different notes at the same time; in others cases like arpeggios and broken chords the notes are played separately and successively. Given a music piece the chords and their progression define the harmonic content of the piece itslef \cite{papadopoulos2007large}. Analyze chord tonality is one of the most important task in both \textit{Chord Recognition} and \textit{Song Identification} \cite{lee2006identifying}.\\
%
The most commonly used chords in music are \textit{triads}, wich coincide with groups of three notes (root, third and perfect fifth). To overcome the issue of managing chords played in different ocatves, it is possible to use \textit{chromas} instead of distinct notes \cite{bartsch2005audio}. Chords can be classified as \textit{minor}, \textit{major}, \textit{dimished} and \textit{augmented}, according to the musical interval of the notes that costituite them. The name of a chord is driven from its root note. Considering triad as example, the second note is called third and the third note is called fifth. The chord is called major (minor) if the third interval is major (minor), i.e. 4 (3) semitones away from the root. All this peculiarities make \textit{Chord Recognition} a very complex task to be achieved. \\
%
Our report is structured as follows. In \ref{sec:techniques} we present a serie of techniques that can be used to extract features from sound data and establish consequently chords. In \ref{sec:setup} we describe two different experiments that we performed. In the first experiment we achieved \textit{Chord Recognition} using sounds containing a single chord produced by a single instrument. In the second experiment we dealt with a much more complex scenario since we used full songs that were composed by several instruments and also human voice. In \ref{sec:results} we describe the obtained results. Finally in \ref{sec:conclusions} we comment the characteristics and the advantages of the different techniques used. 
