\section{Introduction}
\label{sec:intro}

A chord is defined as two or more notes sounding simultaneously. \cite{benward2014music} This is normally achieved by playing the different notes that form the chord at the same time, in other cases like arpeggios and broken chords, the notes are played separated and successively. The harmonic content of a music piece is defined by the chords and the progressions of them. This aspect is very important to understand and analyse tonal music.\cite{papadopoulos2007large} This is an important applications in cover songs identification where the harmonic analysis is more robust.\cite{lee2006identifying}  The chord names vary depending on the number of notes that form them: "Triad","Seventh", "Ninth",... The most commonly used chords in music pieces are the "triads", in which this project focusses. This kind of chord is composed by three different notes. To avoid problems with the different octaves and their pitch, "chromas" are used instead of distinct notes.\cite{bartsch2005audio}\\ The notes of a chord are selected according to the musical distance between pitches of simultaneously sounding notes, the so called musical interval. The chords are commonly classified as "minor", "major", "dimished" and "augmented". This project focusses on minor and major chords. The name of the chord is driven from the root note of it, usually this note is the lowest one. Taking the triad as an example the second note is called third, because it's the third chroma from the root note on, so the last note is called fifth. In the minor triad only the 2 note changes, to a minor third interval. Some chords present inversions, where the root note is not in the lowest position. All this peculiarities form the complexity of the Chord recognition Task.   