\section{Introduction}
\label{sec:intro}

A chord is defined as two or more notes sounding simultaneously \cite{benward2014music}. This is normally achieved by playing the different notes that form the chord at the same time, in other cases like arpeggios and broken chords, the notes are played separately and successively. The harmonic content of a music piece is defined by the chords and their progression. This aspect is very important to understand and analyze tonal music \cite{papadopoulos2007large}. This is an important application in cover songs identification where the harmonic analysis is more robust \cite{lee2006identifying}. The most commonly used chords in music pieces are the \textit{triads}, groups of three notes (root, third and perfect fifth) on which this project focuses. To avoid problems with the different octaves and their pitch, \textit{chromas} are used instead of distinct notes \cite{bartsch2005audio}.

The notes of a chord are selected according to their musical interval. The chords are commonly classified as \textit{minor}, \textit{major}, \textit{dimished} and \textit{augmented}. This project focuses only on minor and major chords. The name of the chord is driven from its root note, usually the lowest note. Taking the triad as an example the second note is called third, because it's the third chroma from the root note and the last note is called fifth. The chord is called major (minor) if the third interval is major (minor), i.e. 4 (3) semitones away from the root. Some chords present inversions, where the root note is not in the lowest position. All this peculiarities form the complexity of the \textit{Chord Recognition Task}.