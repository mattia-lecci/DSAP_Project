\section{Introduction}
\label{sec:intro}

A chord is defined as two or more notes sounding simultaneously \cite{benward2014music}. This is normally achieved by playing the different notes that form the chord at the same time; in others cases like arpeggios and broken chords the notes are played separately and successively. The harmonic content of a music piece is defined by the chords and their progression. The sequence of chords establishes the tonality of the song \cite{papadopoulos2007large}. Therefore the analysis of tonality beacomes the most important topic in song identification \cite{lee2006identifying}. \\
%
The most commonly used chords in music pieces are \textit{triads}, wich coincide with groups of three notes (root, third and perfect fifth). In the first part of this project we focus on triads anlysis, while in the second part we take in account also more complex cases. To avoid the problems caused by the different octaves and their pitch, \textit{chromas} are used instead of distinct notes \cite{bartsch2005audio}.\\
%
The notes that constituite a chord are selected according to their musical interval. The chords are commonly classified as \textit{minor}, \textit{major}, \textit{dimished} and \textit{augmented}. In this project we focus only on minor and major chords. The name of the chord is driven from its root note, usually the lowest note. Considering triad as example the second note is called third, because it's the third chroma from the root note, and the third note is called fifth. The chord is called major (minor) if the third interval is major (minor), i.e. 4 (3) semitones away from the root. We notice that some chords present inversions, if the root note is not in the lowest position. All this peculiarities make \textit{Chord Recognition} a very complex task to be achieved.
